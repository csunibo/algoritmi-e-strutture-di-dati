\documentclass{article}
\usepackage[utf8]{inputenc}
\usepackage{amsmath,amssymb,amsthm}
\usepackage{listings}
\usepackage{enumerate}

\title{\textbf{Strutture dati elementari}}
\author{Luca Tagliavini}
\date{March 8-10, 2021}

\begin{document}

\maketitle
\tableofcontents
\pagebreak

\section{Tipi di dati}

\begin{itemize}
  \item \textbf{Tipo di dato primitivo}: Tipi di dato forniti direttamente dal
    linguaggio e con una diretta rappresentazione a livello macchina con operazioni
    native offerte direttamente dal calcolatore. Esempi possono essere \emph{int, float, double}.
  \item \textbf{Tipo di dato astratto}: Tipo di dato composto e astratto, che combina
    al suo interno vari dati astratti o primitivi e offre un insieme di operazioni da
    svolgere su questi valori. (\emph{astratto rispetto alla sua rappresentazione})
    \begin{itemize}
      \item Specifica: il manuale d'uso, l'interfaccia visibile a chi utilizza
        il nostro codice dall'esterno. Nasconde i dettagli implementativi e
        consente al progettista di esporre solo le operazioni che ritiene giusto.
      \item Implementazione: il codice vero e proprio con cui viene realizzata la
        struttura dati. Non e' visibile allo sviluppatore che utilizza la libreria
        e due diverse implementazioni possono avere una grande disparita' di performance.
    \end{itemize}
\end{itemize}

\subsection{Strutture dati in generale}

I dati sono organizzati in memoria tramite \emph{strutture dati} in modo da oragnizzare
i dati e offrire una serie di operazioni per manipolare la struttura. \\
Esistono diversi metodi di definire strutture dati e le loro interfacce pubbliche, come
ad esempio le interfacce in Java o i template in C++.

\subsubsection{Esempio: Dizionario}

\begin{lstlisting}
public interface Dizionario {
  public void insert(Comparable key, Object value);
  public void delete(Comparable key);
  public Object search(Comparable key);
}
\end{lstlisting}

Decidiamo per esempio di inserire gli elementi come chiavi $(k, v)$ in una array dinamico \textbf{ordinato}.
I nostri metodi faranno dunque le seguenti operazioni:
\begin{itemize}
  \item \textbf{search}: ricerca binaria sull'array ($O(\log_2 n)$)
  \item \textbf{insert}: inserimento ordinato, quindi scorrendo l'array e
    torvando il punto giusto dove inserire e spostare tutti gli elementi
    successivi per fare spazio per il nuovo elemento ($O(n)$)
  \item \textbf{delete}: ricerca binaria dell'elemento e rimozione spostando a
  sinistra tutti gli elementi successivi ($O(n)$)
\end{itemize}

Decidiamo per esempio di inserire gli elementi come chiavi $(k, v)$ in una lista.
I nostri metodi faranno dunque le seguenti operazioni:
\begin{itemize}
  \item \textbf{search}: ricerca sulla lista ($O(n)$)
  \item \textbf{insert}: inserimento a testa ($O(1)$) 
  \item \textbf{delete}: ricecca dell'elemento e cambiamento del puntatore ($O(n)$)
\end{itemize}

Come si puo' notare \emph{varie implementazioni hanno vari tradeoff}.

\subsection{Tipi di strutture dati}

\begin{itemize}
  \item \emph{Lineari}: Elementi in una sequenza dove ogni elemento ha un index (array)
  \item \emph{Non lineari}: Elementi inseriti in una sequenza senza una linearita' in memoria (liste)

  \item \emph{Statiche}: Numero di elementi sempre costante nel tempo
  \item \emph{Dinamiche}: Numero di elementi cambia nel tempo

  \item \emph{Omogenee}: Dati tutti dello stesso tipo
  \item \emph{Non Omogenee}: Dati possono essere di tipo diverso
\end{itemize}

\section{Strutture dati elementari}

\subsection{Strutture concatenate (collegate)}

Due tecniche per rappresentare collezioni di elementi: \\
\textbf{strutture collegate (liste)} oppure \textbf{vettori indicizzati}.
Analizzeremo in particolare le strutture collegate che sono piu' interessati come implementazione:

Tutte le strutture collegate (liste) hanno in comunque di essere una collezione di \textbf{record}:
\begin{itemize}
  \item tutti i record sono collegati tra di loro
  \item i record sono numerabili partendo dalla testa o dalla coda
  \item ogni record contiene un dato
\end{itemize}

\subsubsection{Liste bilinkate}

Per alcuni algoritmi puo' essere svantaggioso avere un riferimento solo al nodo successivo
e si usano dunque liste bilinkate dove ogni record ha un puntatore all'elemento
precedente e a quello successivo. Tuttavia, l'aggiunta di un puntatore occupera' una
word in piu' in memoria per record. Si ha dunque un minimo overhead sulla memoria.

\subsubsection{Liste circolari}

Possiamo oltretutto estendere le liste bilinkate in modo che il nodo precedente
della lista in testa punti all'ultimo e il nodo successivo dell'ultimo record punti
al primo. In questo modo si ha una lista bilinkata circolare che rende molto piu'
veloci operazioni come rimozione in coda o aggiunta in coda.

Chiaramente con questo tipo di liste si puo' avere un problema di guardie per capire
quando si e' in testa o in coda nella lista. (basterebbe controllare quando si analizza
il nodo testa per la seconda volta, al quale punto abbiamo dunque iterato su tutta la lista)

\subsubsection{Liste circolari con nodo sentinella}

Si possono implementare liste circolari (non pulitisse) come sopra ma inserendo in
testa un nodo vuoto che svolge l'unico compito di segnalare quando si e' giunti alla testa dela lista.

\subsection{Stack (o pile)}

Le pile sono una struttura di dati molto diffusa e puo' essere pensata come una
lista dove si aggiunge sempre in testa e si possono rimuovere elementi solo in testa. \\
In questo modo abbiamo un data flow LIFO (Last In, First Out).
Alcuni metodi previsiti per una pila potrebbero essere: 

\begin{itemize}
  \item \textbf{push}: aggiunge un elemento in cima alla lista
  \item \textbf{pop}: rimuove e restituisce l'elemento in cima alla lista
  \item \textbf{top}: restituisce (e non rimuove) il primo elemento sulla lista
  \item \textbf{isEmpty}: restituisce un booleano che indica se la lista e' vuota
\end{itemize}

Alcuni linguaggi come Java Bytecode o Hack VM funzionano con una interfaccia a pile.

Ci sono due principali tecniche di implementazione:
\begin{itemize}
  \item \textbf{liste concatenate}: una lista come descritta sopra dove si
    accede solo all'elemento in testa per i push/pop. \\
    Pro: grandezza infinita \\
    Con: piu' dispendiosa come memoria
  \item \textbf{array}: utilizziamo un array statico di dimensione fissa tenendo
    traccia dell'index dell'ultimo elemento aggiungo (cima della pila). \\
    Pro: meno dispendioso a livello di memoria \\
    Con: grandezza fissa limitata
\end{itemize}

\subsection{Coda (o queue)}

La struttura dati coda, a differenza della pila, segue una politica FIFO (First In, First Out). \\
Le operazioni standard sono:
\begin{itemize}
  \item \textbf{enqueue}: aggiunge un elemento in coda
  \item \textbf{dequeue}: rimuove il primo elemento inserito nella coda
  \item \textbf{first}: restituisce il primo elemento della coda
  \item \textbf{isEmpty}: restituisce un booleano che indica se la lista e' vuota
\end{itemize}

Possibili implementazioni sono:
\begin{itemize}
  \item \textbf{liste}: liste normali o bilinkate, che hanno una performance migliore,
    ma sono decisamente piu' complicate dal punto di vista implementativo. \\
    Serve tenere traccia della $head$ e $tail$ rispettivamente per l'estrazione e l'inserimento.
  \item \textbf{array circolari}: dimensione limitata, hoverhead minore.
\end{itemize}

\section{Analisi ammortizzata}

L'idea e' che se abbiamo una operazione che nel caso peggiore richiede $O(n)$ ma
nel casi normali richiede un $O$ minore, possiamo fare un calcolo ammortizzato,
che tiene conto del tempo risparmiato nelle chiamate meno costose e ci da un risultato
piu' realistico e piu' basso rispetto a quello del caso pessimo.

La tecnica piu' facile e che vederemo e' detta \emph{tecnica dei crediti}. \\
Partiamo dalla definizione di costo artificiale
\begin{align*}
  T^i_{am}(n) = T^i_{ef}(n) + deposito^i(n) - prelievo^i(n)
\end{align*}
Dove $T^i_{am}$ e' il costo ammortizzato e $T^i_{ef}$ e' il costo effettivo. \\
Dimostriamo che:
\begin{align*}
  \sum^k_{i=1} T^i_{am}(n) \geq T(n, k) \\
\end{align*}
Sviluppiamo il primo membro per notare che
\begin{align*}
  \sum^k_{i=1} (T^i_{ef}(n) + deposito^i(n) - prelievo^i(n)) \\
  \sum^k_{i=1} T^i_{ef}(n) + \underline{\sum^k_{i=1} deposito^i(n) - \sum^k_{i=1} prelievo^i(n)}
\end{align*}
la differenza sottolineata e' sempre $\geq 0$ dunque si ha che
\begin{align*}
  \sum^k_{i=1} T^i_{am}(n) \geq \sum^k_{i=1} T^i_{ef}(n) = T(n, k)
\end{align*}

\end{document}
