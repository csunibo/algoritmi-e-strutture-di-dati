\documentclass{article}
\usepackage{amsmath,amssymb,amsthm}
\usepackage{listings}
\usepackage{enumerate}

\title{\textbf{(M, N, K)-game}}
\author{Luca Tagliavini}
\date{March 15-.., 2021}

\begin{document}

\maketitle
\tableofcontents
\pagebreak

\subsection{Informazioni generali}

\begin{itemize}
  \item Si consegna una sola volta. Se necessario saranno richieste modifiche.
  \item Il voto rimane valido anche per gli anni accademici successivi
  \item \textbf{deadline}: Febbraio 2022
  \item il progetto incide $1/3$ sul foto finale
\end{itemize}

\subsection{Steps}

\begin{itemize}
  \item \textbf{sviluppo}: Java. Viene fornita l'interfaccia da implementare
    e software utile per testare \\
    Codici sorgenti zippati, relazione in PDF. Basta che solo una persona
    consegnia i dati. Appelli su AlmaEsami+Teams. La discussione e' subito
    successiva alla consegna(i.e. una settimana).
  \item \textbf{relazione}: Le decisioni prese durante lo sviluppo
    andranno commentate in una breve relazione.
  \item \textbf{discussione orale}: e' circoscritto al progetto.
\end{itemize}

\subsection{Game Tree}

Possiamo rappresentare tutte le possibili partite giocabili di un qualunque gioco
tramite una struttura ad albero. Ogni nodo del nosto albero diventa una possibile
"mappa", (scacchiera per gli scacchi, matrice per l'mnk) del gioco. \\
La partita puo' terminare in \emph{vittoria, sconfitta o patta}.

Il problema con questo approccio e' che l'albero generato e' talmente tanto ampio
che non diventa visitabile in un tempo sensato (i.e. visitare un simile albero per
il gioco degli scacchi richiederebbe piu' tempo della vita attuale dell'universo).

\subsection{Algoritmo minimax}

\begin{itemize}
  \item assumiamo di poter generare tutto l'albero delle possibili combinazioni
  \item aggiungiamo una targetta ad ogni foglia indicando se la foglia corrisponde
    a una vittoria, a una sconfitta o a una patta.
  \item portiamo poi i label dalle foglie alle radici, quindi per i nodi padri
    delle foglie terremo traccia di quante foglie sono vittoriose, quante patte e quante sconfitte.
\end{itemize}

\end{document}
