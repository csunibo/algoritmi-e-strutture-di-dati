\documentclass{article}
\usepackage[utf8]{inputenc}
\usepackage{amsmath,amssymb,amsthm}
\usepackage{listings}
\usepackage{enumerate}

\title{\textbf{Union Find}}
\author{Luca Tagliavini}
\date{April 13, 2021}

\begin{document}

\maketitle
\tableofcontents
\pagebreak

\subsection{Union Find}

Risolve il problema dell'elaborazione di dati su \emph{insiemi disgiunti}.
Immaginiamo di avere due insiemi disgiunti, ossia contenenti elementi che appartengono
solo ad uno o all'altro, matematicamente $U_1 \cap U_2 = \emptyset$. Le operazioni
su questa struttura dati sono le basilari tra due insiemi:

\begin{itemize}
  \item \emph{make\_set}: crea un insieme singoletto a partire da un singolo
    elemento
  \item \emph{find}: cerca tra gli insiemi disponibili quello che contiene il
    dato elemento
  \item \emph{union}: unisce due insiemi tra di loro
\end{itemize}

La struttura dati contiene un insieme dinamico di insiemi
$S = \{ S_1, S_2, \ldots, S_k \}$ tutti disgiunti tra loro. Tutti gli insiemi
complessivamente contengono $n \geq k$ elementi (dove $n$ e' la somma complessiva
di tutti gli elementi degli insiemi, $k$ e' il numero degli insiemi: dato che ogni
insieme e' almeno singoletto l'assunzione e' ovvia).
Oltretutto ogni insieme e' indentificato da un \emph{rappresentante univoco}.

La scelta del rappresentate e' importante, in qunato deve rispettare due leggi:
\begin{itemize}
  \item il rappresentante di $S_i$ deve essere un qualunque valore contenuto in $S_i$
  \item chiamare \emph{find} su $k, K_1 \in S_i$ deve restituire lo stesso rappresentante
  \item il rappresentatnte puo' cambiare solo dopo una operazione di unione
\end{itemize}

Un esempio di problema risolvibile tramite strutture \emph{union find} sono quelli
della gestione delle spedizioni o del tracciamento dei contatti su una PCB.

\subsubsection{Operazioni}

\begin{itemize}
  \item \emph{make\_set(elem $\overline{x}$)} $\rightarrow$ \emph{repr}: creo un insieme
    singoletto $\{ \overline{x}\}$ ed uso il valore $\overline{x}$ come rappresentante univoco. \\
    Notare che per le proprieta' della struttura l'elemento $\overline{x}$ non deve
    essere gia' stato inserito nella struttura dati in precedenza.
  \item \emph{find(elem x)} $\rightarrow$ \emph{repr}: restituice il rappresentante
    dell'insieme che contiene l'elemento $x$.
  \item \emph{union(repr x, repr y)} $\rightarrow$ \emph{repr}: unisce i due insiemi
    indicati da $x$ e $y$, e scegliamo come nuovo rappresentante canonico quello
    di $x$.
\end{itemize}

\subsection{Implementazioni}

\begin{itemize}
  \item \emph{quick-find}: usano alberi di altezza uno per ogni insieme, avendo tempi dell'ordine: \\
    \emph{make\_set}, \emph{find}: $O(1)$, \emph{union}: $O(n)$.
  \item \emph{quick-union}: usano alberi di altezza dinamica per ogni insieme, avendo tempi dell'ordine: \\
    \emph{make\_set}, \emph{union}: $O(1)$, \emph{find}: $O(n)$.
  \item esistono poi implementazioni piu' serie di entrambe le tecniche per avere
    tempi logaritmici sulle operazioni di \emph{find} o \emph{union} che prima erano lineari.
\end{itemize}

\subsubsection{Quick Find}

Usando alberi di altezza uno per gli insiemi, quando si chiama \emph{find(elem x)}
sappiamo che $x$ contiene un riferimento al padre (radice dell'albero), che e'
il rappresentante dell'insieme, quindi il costo sara' chiaramenteo $O(1)$.

Per implementare \emph{union(repr x, repr y)} possiamo invece cambiare semplicemente
i puntatori di tutti i nodi di $y$ in modo che puntino a $x$ come padre. Nel caso
peggiore, in cui uniamo $x = |\{\ldots\}| = 1$ con $y = |\{\ldots\}| = n-1$ e
percio' dovremo fare $n-1$ assegnamenti per i figli di $y$: il costo sara' dunque $O(n)$.

\subsubsection{Quick Union}

Si rappresentano gli insiemi tramite un albero radicato generico, dunque ogni
\emph{union(repr x, repr y)} sara' semplicemente una aggiunta dell'albero $y$ come
ultimo nodo di $x$, il che ha chiaramente costo $O(1)$ (facendo $y.parent = x$).

Per la \emph{find(elem x)} si parte dal nodo $x$ al quale abbiamo un riferimento
e saliamo l'albero fino alla radice la quale contiene il rappresentante, il che
nel caso pessimo (quando l'albero e' lineare come una lista) avra' costo $O(n)$.

\subsubsection{Euristiche di ottimizzazione}

\textbf{Euristiche su Quick Find}:

L'operazione \emph{union} e' poco efficiente nel caso in cui uniamo a un singoletto
un insieme grande $n-1$, ma possiamo fare di meglio in questo caso. Anzi che fare
l'unione classica come abbiamo definito prima, possiamo guardare il "peso" di ogni
albero (mantenuto come valore nella radice in tempo costante $O(1)$ ad ogni operazione)
e quando andiamo a fare l'union svolgiamo il cambiamento dei campi "parent"
sull'insieme che ha peso minore, il che ci garantisce di svolgere meno operazioni
nei casi pessimi. Quando si sposta il primo insieme nel secondo, la specifica sulla
scelta del rappresentante non viene rispettata, quindi aggiorniamo la radice in
modo tale che sia uguale a quella del primo albero.

Prendendo un insieme singoletto contenente $x$, esso potra' al massimo cambiare
padre $\log_2 n$ volte, poiche' ogni volta che viene inserito in un altro
insieme \emph{raddoppia la grandezza dell'insieme di appartenenza di $x$}, e dunque
al piu' il cambiamento sara' necessario al massiom $\log_2 n$ volte (ragionamento
inverso degli algoritmi di ricerca ordinata ad esempio).

Ne segue attraverso alcuni passaggi enunciati nelle slide che il costo nel caso
pessio e' $O(n/2) = O(n)$, tuttavia nel caso medio, facendo un costo ammortizzato
si ha che il costo sara' $O(\log_2 n)$. \\ \\

\textbf{Euristiche su Quick Union};

L'operazione \emph{find} e' particolarmente rallentata nel caso in cui l'albero
assuma una profondita' eccessiva, accomunando la sua struttura a quella di una lista.
Una euristica mirata a risolvere questo problema e' quella che tiene traccia in ogni
radice del \emph{rank} (rango) di un albero che ne indica la profondita' massima.
Come nella euristica precedente grazie a questa metrica possiamo fare scelte
intelligenti durante l'unione di due alberi. In particolare quello che faremo sara'
unire $y$ in $x$ solo se $y.rank < x.rank$, altrimenti faremo l'opposto ricordandoci
di cambiare poi le radici per mantenere la proprieta' del rappresentante.

In questo modo avremo alberi al massimo di profondita' logaritmica. L'operazione
\emph{find} che ha costo pessimo pari alla profondita' massima dell'albero avra'
dunque un costo massimo $O(\log_2 n)$.

\end{document}
